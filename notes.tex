\documentclass[12pt, oneside, letter] {amsart}
\usepackage {amsmath, amssymb}
\renewcommand {\baselinestretch} {1.175}

\theoremstyle {definition}
\newtheorem {thm} {Theorem}
\newtheorem {ax} {AXIOM}

\theoremstyle {remark}
\newtheorem* {defn} {Definition}
\newtheorem* {qn} {Question}
\newtheorem* {note} {Notation}

\oddsidemargin=.25in
\evensidemargin=.25in
\topmargin=0in
\textheight=9in
\textwidth=6.0in

\begin{document}

\pagestyle {plain}
\begin{center}
  Mathematics 551\\
  INTRODUCTION TO TOPOLOGY
\end{center}


The photocopied material presented in this course consists of a list
of axioms, theorems, questions, and definitions, and this course will
be mainly devoted to a development of proofs of these theorems by
members of the class. Each student should attempt to develop such
proofs without consultation with any other person and without any
reference to material in the literature. Such proofs will then be
presented orally in class, and some written proofs may be required.
This procedure is intended to emphasize a very interesting logical
development of the topology of a line, and to enable members of the
class to develop ability and gain experience in proving theorems.

This material will be supplemented with discussions in class, and it
is not intended that it have any priority over other questions and
theorems that are stated in class discussion. Each student should try
to find other questions related to this material.  It is suggested
that each student prepare written proofs of all the theorems and
include them in a notebook.

\vfill\eject \setcounter {page} {1}
\begin{center}
  Linear Point Set Theory
\end{center}

\bigskip

\noindent \textit {Undefined Notions.} The word \underline {point} and
the expression \underline {the point} $x$ \underline {precedes the
  point} y will not be defined. This undefined expression will be
written $x < y$.

\begin{ax}
  $S$ is a collection of points such that
  
  (a) If $x$ and $y$ are different points, then either $x < y$ or $y <
  x$.
  
  (b) If the point $x$ precedes the point y, then $x \neq y$.
  
  (c) If x, y, and $z$ are points such that $x < y$ and $y < z$, then
  $x < z$.
\end{ax}

\begin{thm}
  If $x$ and $y$ are points such that $x < y$, then $y \not< x$.
\end{thm}

\begin{defn}
  If $c$ is a point of a point set $M$ such that no point of $M$
  precedes $c$, then $c$ is called a \underline {first point} of $M$.
  Define \underline {last point} similarly.
\end{defn}

\begin{thm}
  No point set has two first (last) points.
\end{thm}

\begin{thm}
  Every finite point set has a first point and a last point.
\end{thm}

\begin{thm}
  If $M$ is a finite point set consisting of $n$ points, then there
  exist points $a_1, a_2, \dots, a_n$ of $M$ such that $a_1 < a_2 <
  a_3 < \dots < a_n$.
\end{thm}

\begin{defn}
  If $x < z$ and $z < y$, then $z$ is said to be \underline {between}
  $x$ and $y$. The set of all points between $x$ and $y$ is called a
  \underline {region} and will be referred to as the \underline
  {region} $xy$.
\end{defn}

\begin{ax}
  $S$ has no first point and no last point.
\end{ax}

\begin{thm}
  If $x$ is a point, then there is a region containing $x$.
\end{thm}

\begin{qn}
  Is Theorem 5 equivalent to Axiom 2?
\end{qn}

\begin{defn}
  A set $K$ is said to be a \underline {subset} of a set $M$ if every
  element of $K$ is an element of $M$.
\end{defn}

\begin{defn}
  A point $p$ is said to be a \underline {limit point} of a point set
  $M$ if every region containing $p$ contains a point of $M$ distinct
  from $p$.
\end{defn}

\begin{thm}
  If the point set $H$ is a subset of the point set $K$ and $p$ is a
  limit point of $H$, then $p$ is a limit point of $K$.
\end{thm}

\begin{note}
  The intersection of $R_1$ and $R_2$ is denoted by $R_1 \cdot R_2$.
\end{note}

\begin{thm}
  If the point $x$ is common to the regions $R_1$ and $R_2$, then some
  region lies in $R_1 \cdot R_2$ and contains $x$.
\end{thm}

\begin{thm}
  If the point $p$ is a limit point of the sum of two point sets $H$
  and $K$, then $p$ is a limit point of at least one of the sets $H$
  and $K$.
\end{thm}

\begin{thm}
  Extend Theorem 8 to any finite number of point sets.
\end{thm}

\begin{defn}
  Two point sets are said to be \underline {mutually exclusive} if
  they have no point in common.
  
  If $G$ is a collection of point sets such that each pair of them is
  mutually exclusive, then the sets of $G$ are said to be \underline
  {mutually exclusive}.
\end{defn}

\begin{thm}
  If $x$ and $y$ are two points, then there exist two mutually
  exclusive regions, one containing $x$ and the other containing $y$.
  (Hausdorff property)
\end{thm}

\begin{thm}
  No finite point set has a limit point.
\end{thm}

\begin{thm}
  If $p$ is a limit point of the point set $M$, then every region
  containing $p$ contains infinitely many points of $M$.
\end{thm}

\begin{defn}
  A sequence of points $p_1, p_2, p_3, \dots$ is said to \underline
  {converge} to a point $p$ if for every region $R$ containing $p$
  there is a positive integer $k$ such that if $n > k$, $p_n$ lies in
  $ R$.
\end{defn}

\begin{thm}
  No sequence of points converges to each of two points.
\end{thm}

\begin{thm}
  If $a$ is a sequence of points converging to a point $p$, then every
  subsequence of $a$ converges to $p$.
\end{thm}

\begin{thm}
  If $p_1, p_2, p_3, \dots$ is a sequence of distinct points
  converging to the point $p$, then $p$ is the only limit point of the
  set \linebreak \mbox {$p_1 + p_2 + p_3 + \dots$}
\end{thm}

\begin{thm}
  If $p_1, p_2, p_3, \dots$ is a sequence of points converging to a
  point $p$, then some region contains the set $p + p_1 + p_2 + p_3 +
  \dots$
\end{thm}

\begin{defn}
  A point set is said to be \underline {closed} if it contains all of
  its limit points.
\end{defn}

\begin{note}
  If $H$ is a point set, then $\overline H$ denotes $H$ together with
  all of its limit points. The set $\overline H$ is called the closure
  of $H$.
\end{note}

\begin{thm}
  If $H$ is a point set, then $\overline H$ is closed.
\end{thm}

\begin{defn}
  A point set $H$ is said to be \underline {open} if for every point
  $x$ of $H$ there is a region containing $x$ and lying in $H$.
\end{defn}

\begin{thm}
  If $x$ is a point, then the set of all points which precede (follow)
  $x$ is an open set.
\end{thm}

\begin{thm}
  Every region is an open set.
\end{thm}

\begin{defn}
  If $x$ and $y$ are two points, the set consisting of $x$ and $y$,
  together with all points between $x$ and $y$, is called the
  \underline {interval} $xy$.
\end{defn}

\begin{thm}
  Every interval is closed.
\end{thm}

\begin{defn}
  A set $K$ is said to be a \underline {proper subset} of a set $M$ if
  $K$ is a subset of $M$ and does not contain $M$.
\end{defn}

\begin{thm}
  If the open set $M$ is a proper subset of $S$, then $S - M$ is
  closed.
\end{thm}

\begin{thm}
  If the closed set M is a proper subset of $S$, then $S - M$ is open.
\end{thm}

\begin{note}
  If $G$ is a collection of point sets, then $G^*$ denotes the sum of
  the sets of $G$.
\end{note}

\begin{thm}
  If $G$ is a finite collection of closed sets, then $G^*$ is closed.
\end{thm}

\begin{thm}
  If $G$ is a collection of open sets, then $G^*$ is open.
\end{thm}

\begin{defn}
  The \underline {intersection} of the elements of a collection $G$ of
  point sets is the set of all points that belong to every element of
  $G$.
\end{defn}

\begin{thm}
  If $G$ is a collection of closed sets having a common point, then
  the intersection of the sets of $G$ is closed.
\end{thm}

\begin{thm}
  If $G$ is a finite collection of open sets having a common point,
  then the intersection of the sets of $G$ is open.
\end{thm}

\begin{defn}
  Two sets are said to be \underline {mutually separated} if they have
  no point in common and neither of them contains a limit point of the
  other.
\end{defn}

\begin{defn}
  A point set is said to be \underline {connected} if it is not the
  sum of two mutually separated point sets.
\end{defn}

\begin{thm}
  If $H$ and $K$ are two mutually separated point sets and $M$ is a
  connected subset of $H + K$, then $M$ is a subset of one of the sets
  $H$ and $K$.
\end{thm}

\begin{thm}
  Every point is a connected set.
\end{thm}

\begin{thm}
  If a finite point set $M$ contains more than one point, then $M$ is
  not connected.
\end{thm}

\begin{thm}
  If $G$ is a collection of connected point sets having a common
  point, then $G^*$ is connected.
\end{thm}

\begin{thm}
  If $M$ is a connected point set, then $\overline M$ is connected.
\end{thm}

\begin{defn}
  A point set is said to be \underline {nondegenerate} if it contains
  more than one point.
\end{defn}

\begin{thm}
  If $M$ is a nondegenerate connected point set, then every point of
  $M$ is a limit point of $M$.
\end{thm}

\begin{thm}
  If $x$ and $y$ are two points of a connected point set $M$, then the
  region $xy$ lies in $M$.
\end{thm}

\begin{thm}
  If the point set $M$ is not connected, then $M$ is the sum of two
  mutually separated sets $H$ and $K$ such that every point of $H$
  precedes every point of $K$. (Dedekind cut)
\end{thm}

\begin{ax}
  $S$ is connected.
\end{ax}

\begin{thm}
  If $H$ and $K$ are two point sets such that $H + K$ = $S$ and every
  point of $H$ precedes every point of $K$, then either (1) $H$ has a
  last point or (2) $K$ has a first point. Furthermore, (1) and (2)
  are mutually exclusive.
\end{thm}

\begin{thm}
  Every interval is connected.
\end{thm}

\begin{thm}
  Every region is connected.
\end{thm}

\begin{thm}
  No proper subset of $S$ is both open and closed.
\end{thm}

\begin{qn}
  Is each of the above four theorems equivalent to Axiom 3?
\end{qn}

\begin{thm}
  If $x$ and $y$ are two points, there is a point between them.
\end{thm}

\begin{thm}
  Every point of a region $R$ is a limit point of $R$.
\end{thm}

\begin{thm}
  Each end point of a region $R$ is a limit point of $R$.
\end{thm}

\begin{thm}
  No region has a first (last) point.
\end{thm}

\begin{thm}
  $S$ contains infinitely many points and every point is a limit point
  of $S$.
\end{thm}

\begin{defn}
  A sequence of points is said to be \underline {bounded} if some
  region contains every point of this sequence.
\end{defn}

\begin{defn}
  A sequence of points $p_1, p_2, p_3, \dots$ is said to be \underline
  {increasing} if for each positive integer $n$, $p_n < p_{n+1}$.
\end{defn}

\begin{thm}
  Every bounded increasing sequence of points converges.
\end{thm}

\begin{thm}
  If $a$ is a bounded sequence of points, then some subsequence of $a$
  converges. (Sequential compactness)
\end{thm}

\begin{defn}
  A point set is said to be \underline {bounded} if it lies in some
  region.
\end{defn}

\begin{thm}
  Every bounded infinite point set has a limit point.
  (Bolzano-Weierstrass)
\end{thm}

\begin{thm}
  Every closed and bounded point set contains a first point and a last
  point.
\end{thm}

\begin{thm}
  If $M_1, M_2, M_3, \dots$ is a sequence of closed and bounded point
  sets such that for each $n$, $M_n$ contains $M_{n+1}$, then the sets
  $M_1, M_2, M_3, \dots$ have a point in common. Furthermore, the set
  of all points common to these sets is closed.
\end{thm}

\begin{defn}
  A collection $G$ of point sets is said to \underline {cover} a point
  set $M$ if every point of $M$ lies in at least one set of $G$.
\end{defn}

\begin{thm}
  If $ab$ is a region and $G$ is a collection of regions covering
  $\overline {ab}$, then some finite subcollection of $G$ covers
  $\overline{ab}$.
\end{thm}

\begin{thm}
  Strengthen Theorem 49 by letting $G$ be a collection of open sets.
\end{thm}

\begin{thm}
  If $G$ is a collection of open sets covering the closed and bounded
  point set $M$, then some finite subcollection of $G$ covers $M$.
  (Heine-Borel-Lebesgue)
\end{thm}

\begin{defn}
  A point set $M$ is said to be \underline {perfect} if $M$ is closed
  and every point of $M$ is a limit point of $M$.
\end{defn}

\begin{thm}
  No countable point set is perfect. (Baire)
\end{thm}

\begin{thm}
  Every region is uncountable.
\end{thm}

\begin{thm}
  Every uncountable point set has a limit point.
\end{thm}

\begin{defn}
  A point set $M$ is said to be \underline {nowhere dense in} $S$ if
  every region contains a region which does not intersect $M$.
\end{defn}

\begin{defn}
  If $H$ and $K$ are point sets, $H$ is \underline {nowhere dense in}
  $K$ if every region intersecting $K$ contains a region which
  intersects $K$ but not $H$.
\end{defn}

\begin{thm}
  If the point set $H$ is nowhere dense in the point set $K$, then
  every subset of $H$ is nowhere dense in $K$.
\end{thm}

\begin{thm}
  If each of the point sets $H$ and $K$ is nowhere dense in the point
  set $M$, then $H + K$ is nowhere dense in $M$.
\end{thm}

\begin{thm}
  No region is the sum of a countable number of point sets such that
  each of them is nowhere dense in $S$. (Baire)
\end{thm}

\begin{thm}
  No closed point set $M$ is the sum of a countable number of closed
  point set such that if $x$ is any one of them, then every point of
  $X$ is a limit point of $M - X$. (Baire)
\end{thm}

\begin{thm}
  No closed point set $M$ is the sum of a countable number of point
  sets such that each of them is nowhere dense in $S$. (Baire)
\end{thm}

\begin{defn}
  A point set $H$ is said to be \underline {everywhere dense} in a
  point set $K$ if $\overline H$ contains $K$.
\end{defn}

\begin{thm}
  If the point set $H$ is everywhere dense in the point set $K$, then
  every region intersecting $K$ contains a point of $H$.
\end{thm}

\begin{thm}
  If the point set $H$ is everywhere dense in the point set $K$, then
  $H$ fails to be nowhere dense in $K$.
\end{thm}

\begin{thm}
  If the point set $H$ is nowhere dense in the closed point set $M$,
  then $M - H$ is everywhere dense in $M$.
\end{thm}

\begin{thm}
  If $G$ is a countable collection of open subsets of the point set
  $M$, each of which is everywhere dense in $M$, then the intersection
  of the elements of $G$ is everywhere dense in $M$.
\end{thm}

\begin{defn}
  If the point set $M$ contains a countable set which is everywhere
  dense in $M$, then $M$ is said to be \underline {separable}.
\end{defn}

\begin{ax}
  $S$ is separable.
\end{ax}

\begin{thm}
  There do not exist uncountable many mutually exclusive regions.
\end{thm}

\begin{qn}
  Is Theorem 64 equivalent to Axiom 4? (Souslin
  question---undecidable)
\end{qn}

\begin{thm}
  There exists an increasing unbounded sequence.
\end{thm}

\begin{thm}
  If $p$ is a point, there exists an increasing sequence converging to
  $p$.
\end{thm}

\begin{thm}
  If $p$ is a limit point of the point set $M$, there exists a
  sequence of distinct points in $M$ converging to $p$.
\end{thm}

\begin{thm}
  There exists a countable collection $F$ of regions such that if $R$
  is a region and $p$ is a point of $R$, then some region of $F$
  contains $p$ and lies in $R$. (countable basis---second
  countability---perfectly separable)
\end{thm}

\begin{qn}
  Is Theorem 68 equivalent to Axiom 4?
\end{qn}

\begin{thm}
  If $G$ is a collection of regions covering the point set $M$, then
  some countable subcollection of $G$ covers $M$. (Lindel\" {o}f)
\end{thm}

\begin{thm}
  Every uncountable point set contains a limit point of itself.
\end{thm}

\begin{thm}
  Every point set is separable.
\end{thm}

\begin{thm}
  There does not exist a collection of mutually exclusive intervals
  which covers $S$. (Baire)
\end{thm}

\newpage

\begin{center}
  Transformations
\end{center}

\begin{defn}
  A \underline {transformation} $f$ of a point set $H$ onto a point
  set $K$ is a collection of ordered pairs of points such that:
  \begin{enumerate}
  \item Each ordered pair in $f$ has a point of $H$ as its first
    element and a point of $K$ as its second element.
  \item Every point of $H$ is the first element of some ordered pair
    in $f$ and every point of $K$ is the second element of some
    ordered pair in $f$.
  \item No two pairs in $f$ have the same first element.
  \end{enumerate}
\end{defn}

 For the following definitions and notation, let $f$ be a
 transformation of a point set $H$ onto a point set $K$.
 
 \begin{defn}
   If $K$ is a subset of $K^\prime$, then f is said to be a transformation
   of $H$ into $K^\prime$.
 \end{defn}

 \begin{note}   
   If $x$ is a point of $H$, then $f(x)$ denotes the second element of
   the ordered pair of $f$ that has $x$ as a first element.
   
   If $H^\prime$ is a subset of $H$, then $f(H^\prime)$ denotes the set of all
   points $f(x)$ such that $x$ is a point of $H^\prime$.
   
   If $y$ is a point of $K$, then $f^{-1}(y)$ denotes the set of all
   points $x$ in $H$ such that $f(x) = y$.
   
   If $K^\prime$ is a subset of $K$, then $f^{-1}(K^\prime)$ denotes the set
   of all points in $H$ such that $f(x)$ is a point of $K^\prime$.
 \end{note}

 \begin{defn}
   The transformation $f$ is said to be \underline {one-to-one} if no
   two ordered pairs in $f$ have the same second element.
   
   The transformation f is said to be \underline {continuous} if for
   any sequence $x_1, x_2, x_3, \dots$ of points in $H$ converging to
   a point $x$ in $H$, the sequence $f(x_1), f(x_2), f(x_3), \dots$
   converges to $f(x)$.
\end{defn}

\begin{note} 
  If the transformation $f$ is one-to-one, then $f^{-1}$ denotes the
  collection of all ordered pairs $\left(y, f^{-1}(y)\right)$ such
  that $y$ is a point of $K$.
\end{note}

\begin{defn}
  If the transformation $f$ is one-to-one and both $f$ and $f^{-1}$
  are continuous, then $f$ is said to be a \underline {homeomorphism}
  of $H$ onto $K$, and $H$ and $K$ are said to be \underline
  {homeomorphic}.
\end{defn}

\begin{thm}
  If $f$ is a continuous transformation of an interval $H$ into
  itself, then there is a point $x$ of $H$ such that $f(x) = x$. (H is
  in a space satisfying Axioms 1, 2, 3, 4.)
\end{thm}

\begin{thm}
  If $f$ is a continuous transformation from a point set $H$ into a
  point set $K$, then:

  \begin{enumerate}
  \item If $H$ is closed and bounded, then $K$ is closed and bounded.
  \item If $H$ is connected, then $K$ is connected.
  \item If $H$ is an interval, then $K$ is either an interval or a
    point.
  \end{enumerate}
\end{thm}

\begin{defn}
  A transformation $f$ of a space $S_1$ onto a space $S_2$ is said to
  \underline {preserve order} if for any two points $x$ and $y$ in
  $S_1$ such that if $x < y$, then $f(x) < f(y)$.
  
  Define \underline{reverse order} similarly.
\end{defn}

\begin{thm}
  If $S_1$ and $S_2$ are spaces satisfying Axioms 1, 2, 3, and 4 and
  $f$ is an order preserving (reversing) transformation of $S_1$ onto
  $S_2$, then $f$ is a homeomorphism.
\end{thm}

\begin{thm}
  If $S_1$ and $S_2$ are spaces satisfying Axioms 1, 2, 3, and 4 and
  $f$ is a homeomorphism of $S_1$ and $S_2$, then $f$ either preserves
  order or reverses order.
\end{thm}

\begin{thm}
  Every two spaces satisfying Axioms 1, 2, 3, and 4 are homeomorphic.
\end{thm}

\noindent \textbf {AXIOM 2$^\prime$.} $S$ is nondegenerate and has a first
 point and a last point.

Note: It should be observed here that with the previous definition of
a region, the first (last) point of $S$ does not lie in a region. In
addition to the sets previously defined as regions, consider the
following sets as regions: For each point $p$ different from the first
(last) point of $S$, let the set of all points that precede (follow)
$p$ be a region.

\begin{thm}
  Every two spaces satisfying Axioms 1, 2$^\prime$, 3, and 4 are
  homeomorphic.
\end{thm}

\end{document}

